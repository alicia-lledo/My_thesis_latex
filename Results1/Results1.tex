\chapter{Establishment of laboratory methods and amalytical tools to assay genome-wide chromatin accessibility in clinical samples}
\label{ch:Results 1}


%%%%%%%%%%%%%%%%%%%%%%%%%%%%%%%%%%%%%%%%%%%%%%%%%%
\section{Introduction}
\subsection*{Previous and current methods to identify the accessible genome in cells and tissues}
\subsection*{Implementation of ATAC-seq to define the chromatin landscape}
\subsection*{Technical limitations and recent advances in optimisation}
\subsection*{Challenges of working with clinical samples}

%%%%%%%%%%%%%%%%%%%%%%%%%%%%%%%%%%%%%%%%%%%%%%%%%%
\section{Results}
%

\subsection*{Implementation of an ATAC-seq data analysis pipeline based on current knowledge}
When the first ATAC-seq publication by Buenrostro \textit{et al.}, 2013 appeared there were not well established protocols for the processing of the data. Since then, several publications have used ATAC-seq and modifications of this protocol to answer different biological questions and have been implementing different ways of analysing the data. 

*Mention main limitations


*Table summary of how different papers have addressed it



*Choosing the way to proceed for each of those limiting steps and explaining with a graph why
	* Assessment of QC: do TSS for the CD4
	* Peak calling
	* Differential analysis: Data to use maybe the core data CD14 vs CD4
	
	
	
	
\begin{landscape}
\begin{table}[htbp]
\setlength{\tabcolsep}{20pt}
%\renewcommand{\arraystretch}{1.5} makes it longer
\begin{center}
\begin{tabular}{@{} c c}
\toprule
\textbf{Publication} & \textbf{Peak calling and filtering} & \textbf{Master list} & \textbf{Differential analysis} \\
\midrule
\midrule
Corces \textit{et al.,}2016 & MACS2 --nomodel  & Rank peak summits -log10pval & ??Quantile normalisation of count matrix\\
                          & Peak summit extension +/-250bp 									& Non overlapping peaks maximally significant & In-house Pearson correlation method\\
\midrule
ENCODE 										& MACS2 --nomodel  								& All IDR significant peaks & Not established\\
                          & Pseudorreplicates IDR analysis 	&  													& \\
													& Only peaks with IDR X 					&  													& \\

\midrule
Matthias 									& MACS2 --nomodel  								& All IDR significant peaks & Not established\\
                          & Pseudorreplicates IDR analysis 	&  													& \\
													& Only peaks with IDR X 					&  													& \\


\midrule
Alasoo \textit{et al.,}2017 & MACS2 --nomodel  								& All IDR significant peaks & Not established\\
														& Pseudorreplicates IDR analysis 	&  													& \\
														& Only peaks with IDR X 					&  													& \\


\midrule
Qu \textit{et al.,}2017 		& MACS2 --nomodel  									& All IDR significant peaks & Not established\\
														& Pseudorreplicates IDR analysis 		&  													& \\
														& Only peaks with IDR X 						&  													& \\

Rendeiro \textit{et al.,}2016 & MACS2 --nomodel  								& All IDR significant peaks & Not established\\
															& Pseudorreplicates IDR analysis 	&  													& \\
															& Only peaks with IDR X 					&  													& \\													
\bottomrule
\end{tabular}
\medskip %gap
\caption[Summary table of ATAC-seq methodology analysis for peak calling, filtering and differential analysis]{\textbf{.}}
\label{tab:ATAC_comparative_methods}
\end{center}
\end{table}
\end{landscape}
\bigskip %bigger space

