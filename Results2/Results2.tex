\chapter{Defining the genome-wide chromatin accessibility landscape and gene expression profile in psoriasis}
\label{ch:Results 2}


%%%%%%%%%%%%%%%%%%%%%%%%%%%%%%%%%%%%%%%%%%%%%%%%%%
\section{Introduction}
%

%%%%%%%%%%%%%%%%%%%%%%%%%%%%%%%%%%%%%%%%%%%%%%%%%%
\section{Results}
%

\subsection*{Description of skin biopsies derived sample cohort}
\subsubsection{Sample cohort and ATAC-seq approaches}
The skin, more precisely the epidermis, is the primary organ affected by PS. The epidermis is mainly formed by KC at different differentiation stages depending on the layer where they are located.  The KC  play a key role in the dysregulation of the skin renewal in disease and have been considered immune cells actively involved in the chronic inflammation at this tissue.\par
There is DHS publicly available data generated using cultured primary normal human epidermal keratinocytes (NHEK) which provide a general map for chromatin accessibility in this cell type. However, it has been well described that monolayer cultured KC show alteration of physiological characteristics, some of them due to absence of 3D structure, contact with other cell types or the composition of the growth media. These lead, among others, to an increase in the differentiation rate and loss of proliferative ability. There is also evidence of the difficulty to mimic the \textbf{\textit{in vivo}} PS cytokine mileu leading to noticeable differences in the transcriptomic profiles between patients' KC isolated from lesional skin and primary KC stimulated in culture.\par The reduction in cell numbers and experimental time offered by the ATAC-seq protocol allowed to attempt the characterisation of the chromatin accessibility landscape of controls and paired uninvolved and lesional skin biopsies directly isolated from PS patients. Different procedures for cell isolation and modification of the ATAC-seq protocol were assayed in a cohort of control and PS skin biopsies (Table \ref{tab:SkinCohort}) 

\begin{table}[htbp]
\begin{tabular}{@{} c c c c c}
\toprule
\textbf{Sample ID} & \textbf{Phenotype}  & \textbf{Uninvolved}  & \textbf{Lesional} & \textbf{Experimental approach} \\
\midrule
PS8000        & PsV & Yes & Yes & IA \\
PS899         & PsV & No & Yes & IB \\
PS2016        & PsV & Yes & Yes & II \\
SB_CTL1       & Control & NA & NA & II \\
SB_CTL2       & Control & NA & NA & II \\
SB_CTL3       & Control & NA & NA & III (4 different conditions) \\
SB_CTL4       & Control & NA & NA & III\\
SB_CTL5       & Control & NA & NA & III\\
SB_CTL6       & Control & NA & NA & III\\
PS2000B       & PsV & Yes & Yes & III\\
PS2001B       & PsV & Yes & Yes & III\\
\bottomrule
\end{tabular}
\medskip %gap
\caption[Skin biopsies cohort and experimental approach to study the chromatin accessibility landscape using ATAC-seq]{\textbf{Basic information for the different skin biopsies collected from PsV patients or control individuals. For the patients, information regarding availability of uninvolved and lesional biopsy sample is specified. Each sample was used to assay a particular experimental approach coded as IA, IB, II and III}}
\label{tab:SkinCohort}
\end{table}
\bigskip %bigger space


The initial steps of the skin biopsy processing (including removal of adipose tissue, overnight dispase digestion, epidermis peeling and trypsin digestion) were shared across all the approaches. The different approaches to perform ATAC-seq differ in the targeted isolated cells and variations in the ATAC-seq protocol, particularly lysis and time of transposition.


\begin{table}[htbp]
\begin{tabular}{@{} c c c c c}
\toprule
\textbf{Approach} & \textbf{Targeted cells}  & \textbf{Cell number}  & \textbf{Adherence time (min)} & \textbf{ATAC-seq protocol} & \textbf{Transposition time (min)} \\
\midrule
IA        & PsV & Yes & Yes & IA \\
IB        & PsV & No & Yes & IB \\
II       & PsV & Yes & Yes & II \\
III       & Control & NA & NA & II \\
\bottomrule
\end{tabular}
\medskip %gap
\caption[Skin biopsies cohort and experimental approach to study the chromatin accessibility landscape using ATAC-seq]{\textbf{Basic information for the different skin biopsies collected from PsV patients or control individuals. For the patients, information regarding availability of uninvolved and lesional biopsy sample is specified. Each sample was used to assay a particular experimental approach coded as IA, IB, II and III}}
\label{tab:SkinCohort}
\end{table}
\bigskip %bigger space








%%%%%%%%%%%%%%%%%%%%%%%%%%%%%%%%%%%%%%%%%%%%%%%%%%
\section{Discussion}
%






